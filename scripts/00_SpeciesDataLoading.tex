% Options for packages loaded elsewhere
\PassOptionsToPackage{unicode}{hyperref}
\PassOptionsToPackage{hyphens}{url}
%
\documentclass[
]{article}
\usepackage{amsmath,amssymb}
\usepackage{lmodern}
\usepackage{iftex}
\ifPDFTeX
  \usepackage[T1]{fontenc}
  \usepackage[utf8]{inputenc}
  \usepackage{textcomp} % provide euro and other symbols
\else % if luatex or xetex
  \usepackage{unicode-math}
  \defaultfontfeatures{Scale=MatchLowercase}
  \defaultfontfeatures[\rmfamily]{Ligatures=TeX,Scale=1}
\fi
% Use upquote if available, for straight quotes in verbatim environments
\IfFileExists{upquote.sty}{\usepackage{upquote}}{}
\IfFileExists{microtype.sty}{% use microtype if available
  \usepackage[]{microtype}
  \UseMicrotypeSet[protrusion]{basicmath} % disable protrusion for tt fonts
}{}
\makeatletter
\@ifundefined{KOMAClassName}{% if non-KOMA class
  \IfFileExists{parskip.sty}{%
    \usepackage{parskip}
  }{% else
    \setlength{\parindent}{0pt}
    \setlength{\parskip}{6pt plus 2pt minus 1pt}}
}{% if KOMA class
  \KOMAoptions{parskip=half}}
\makeatother
\usepackage{xcolor}
\usepackage[margin=1in]{geometry}
\usepackage{color}
\usepackage{fancyvrb}
\newcommand{\VerbBar}{|}
\newcommand{\VERB}{\Verb[commandchars=\\\{\}]}
\DefineVerbatimEnvironment{Highlighting}{Verbatim}{commandchars=\\\{\}}
% Add ',fontsize=\small' for more characters per line
\usepackage{framed}
\definecolor{shadecolor}{RGB}{248,248,248}
\newenvironment{Shaded}{\begin{snugshade}}{\end{snugshade}}
\newcommand{\AlertTok}[1]{\textcolor[rgb]{0.94,0.16,0.16}{#1}}
\newcommand{\AnnotationTok}[1]{\textcolor[rgb]{0.56,0.35,0.01}{\textbf{\textit{#1}}}}
\newcommand{\AttributeTok}[1]{\textcolor[rgb]{0.77,0.63,0.00}{#1}}
\newcommand{\BaseNTok}[1]{\textcolor[rgb]{0.00,0.00,0.81}{#1}}
\newcommand{\BuiltInTok}[1]{#1}
\newcommand{\CharTok}[1]{\textcolor[rgb]{0.31,0.60,0.02}{#1}}
\newcommand{\CommentTok}[1]{\textcolor[rgb]{0.56,0.35,0.01}{\textit{#1}}}
\newcommand{\CommentVarTok}[1]{\textcolor[rgb]{0.56,0.35,0.01}{\textbf{\textit{#1}}}}
\newcommand{\ConstantTok}[1]{\textcolor[rgb]{0.00,0.00,0.00}{#1}}
\newcommand{\ControlFlowTok}[1]{\textcolor[rgb]{0.13,0.29,0.53}{\textbf{#1}}}
\newcommand{\DataTypeTok}[1]{\textcolor[rgb]{0.13,0.29,0.53}{#1}}
\newcommand{\DecValTok}[1]{\textcolor[rgb]{0.00,0.00,0.81}{#1}}
\newcommand{\DocumentationTok}[1]{\textcolor[rgb]{0.56,0.35,0.01}{\textbf{\textit{#1}}}}
\newcommand{\ErrorTok}[1]{\textcolor[rgb]{0.64,0.00,0.00}{\textbf{#1}}}
\newcommand{\ExtensionTok}[1]{#1}
\newcommand{\FloatTok}[1]{\textcolor[rgb]{0.00,0.00,0.81}{#1}}
\newcommand{\FunctionTok}[1]{\textcolor[rgb]{0.00,0.00,0.00}{#1}}
\newcommand{\ImportTok}[1]{#1}
\newcommand{\InformationTok}[1]{\textcolor[rgb]{0.56,0.35,0.01}{\textbf{\textit{#1}}}}
\newcommand{\KeywordTok}[1]{\textcolor[rgb]{0.13,0.29,0.53}{\textbf{#1}}}
\newcommand{\NormalTok}[1]{#1}
\newcommand{\OperatorTok}[1]{\textcolor[rgb]{0.81,0.36,0.00}{\textbf{#1}}}
\newcommand{\OtherTok}[1]{\textcolor[rgb]{0.56,0.35,0.01}{#1}}
\newcommand{\PreprocessorTok}[1]{\textcolor[rgb]{0.56,0.35,0.01}{\textit{#1}}}
\newcommand{\RegionMarkerTok}[1]{#1}
\newcommand{\SpecialCharTok}[1]{\textcolor[rgb]{0.00,0.00,0.00}{#1}}
\newcommand{\SpecialStringTok}[1]{\textcolor[rgb]{0.31,0.60,0.02}{#1}}
\newcommand{\StringTok}[1]{\textcolor[rgb]{0.31,0.60,0.02}{#1}}
\newcommand{\VariableTok}[1]{\textcolor[rgb]{0.00,0.00,0.00}{#1}}
\newcommand{\VerbatimStringTok}[1]{\textcolor[rgb]{0.31,0.60,0.02}{#1}}
\newcommand{\WarningTok}[1]{\textcolor[rgb]{0.56,0.35,0.01}{\textbf{\textit{#1}}}}
\usepackage{graphicx}
\makeatletter
\def\maxwidth{\ifdim\Gin@nat@width>\linewidth\linewidth\else\Gin@nat@width\fi}
\def\maxheight{\ifdim\Gin@nat@height>\textheight\textheight\else\Gin@nat@height\fi}
\makeatother
% Scale images if necessary, so that they will not overflow the page
% margins by default, and it is still possible to overwrite the defaults
% using explicit options in \includegraphics[width, height, ...]{}
\setkeys{Gin}{width=\maxwidth,height=\maxheight,keepaspectratio}
% Set default figure placement to htbp
\makeatletter
\def\fps@figure{htbp}
\makeatother
\setlength{\emergencystretch}{3em} % prevent overfull lines
\providecommand{\tightlist}{%
  \setlength{\itemsep}{0pt}\setlength{\parskip}{0pt}}
\setcounter{secnumdepth}{-\maxdimen} % remove section numbering
\ifLuaTeX
  \usepackage{selnolig}  % disable illegal ligatures
\fi
\IfFileExists{bookmark.sty}{\usepackage{bookmark}}{\usepackage{hyperref}}
\IfFileExists{xurl.sty}{\usepackage{xurl}}{} % add URL line breaks if available
\urlstyle{same} % disable monospaced font for URLs
\hypersetup{
  pdftitle={00\_SpeciesDataLoading},
  pdfauthor={Madrone Environmental Services},
  hidelinks,
  pdfcreator={LaTeX via pandoc}}

\title{00\_SpeciesDataLoading}
\author{Madrone Environmental Services}
\date{2023-01-19}

\begin{document}
\maketitle

\begin{Shaded}
\begin{Highlighting}[]
\CommentTok{\#This bit of code turns off warning messages in your R markdown output.}
\NormalTok{knitr}\SpecialCharTok{::}\NormalTok{opts\_chunk}\SpecialCharTok{$}\FunctionTok{set}\NormalTok{(}\AttributeTok{warning =} \ConstantTok{FALSE}\NormalTok{, }\AttributeTok{message =} \ConstantTok{FALSE}\NormalTok{)}
\end{Highlighting}
\end{Shaded}

\hypertarget{standardize-your-libraries-for-use-for-the-modelling-process.}{%
\subsection{Standardize your libraries for use for the modelling
process.}\label{standardize-your-libraries-for-use-for-the-modelling-process.}}

\hypertarget{loading-data-from-gbif}{%
\subsection{Loading Data from GBIF}\label{loading-data-from-gbif}}

Use this script to load data from Global Biodiversity Information
Framework (GBIF):

\begin{Shaded}
\begin{Highlighting}[]
\NormalTok{myspecies }\OtherTok{\textless{}{-}} \FunctionTok{c}\NormalTok{(}\StringTok{"Plestiodon skiltonianus"}\NormalTok{)}

\CommentTok{\# download GBIF occurrence data for this species; this takes time if there are many data points!}
\NormalTok{gbif\_data }\OtherTok{\textless{}{-}} \FunctionTok{occ\_data}\NormalTok{(}\AttributeTok{scientificName =}\NormalTok{ myspecies, }\AttributeTok{hasCoordinate =} \ConstantTok{TRUE}\NormalTok{, }\AttributeTok{limit =} \DecValTok{20000}\NormalTok{, }\AttributeTok{decimalLongitude =} \StringTok{"{-}124, {-}114"}\NormalTok{, }\AttributeTok{decimalLatitude =} \StringTok{"49, 50"}\NormalTok{)}

\CommentTok{\# take a look at the downloaded data:}
\NormalTok{gbif\_data}
\end{Highlighting}
\end{Shaded}

\begin{verbatim}
## Records found [61] 
## Records returned [61] 
## Args [hasCoordinate=TRUE, decimalLatitude=49, 50, decimalLongitude=-124, -114,
##      occurrenceStatus=PRESENT, limit=20000, offset=0, scientificName=Plestiodon
##      skiltonianus] 
## # A tibble: 61 x 107
##    key        scientificName  decimalLatitude decimalLongitude issues datasetKey
##    <chr>      <chr>                     <dbl>            <dbl> <chr>  <chr>     
##  1 3760409541 Plestiodon ski~            49.2            -118. cdc,c~ 50c9509d-~
##  2 3859475795 Plestiodon ski~            49.1            -120. cdc,c~ 50c9509d-~
##  3 3947544710 Plestiodon ski~            49.3            -119. cdc,c~ 50c9509d-~
##  4 3873184905 Plestiodon ski~            49.5            -117. cdc,c~ 50c9509d-~
##  5 3902675359 Plestiodon ski~            49.5            -120. cdc,c~ 50c9509d-~
##  6 3903369741 Plestiodon ski~            49.4            -118. cdc,c~ 50c9509d-~
##  7 3902518370 Plestiodon ski~            49.1            -118. cdc,c~ 50c9509d-~
##  8 3902382094 Plestiodon ski~            49.1            -118. cdc,c~ 50c9509d-~
##  9 3903318078 Plestiodon ski~            49.0            -118. cdc,c~ 50c9509d-~
## 10 3903399939 Plestiodon ski~            49.1            -118. cdc,c~ 50c9509d-~
## # ... with 51 more rows, and 101 more variables: publishingOrgKey <chr>,
## #   installationKey <chr>, publishingCountry <chr>, protocol <chr>,
## #   lastCrawled <chr>, lastParsed <chr>, crawlId <int>,
## #   hostingOrganizationKey <chr>, basisOfRecord <chr>, occurrenceStatus <chr>,
## #   taxonKey <int>, kingdomKey <int>, phylumKey <int>, classKey <int>,
## #   familyKey <int>, genusKey <int>, speciesKey <int>, acceptedTaxonKey <int>,
## #   acceptedScientificName <chr>, kingdom <chr>, phylum <chr>, ...
\end{verbatim}

\begin{Shaded}
\begin{Highlighting}[]
\NormalTok{myspecies\_coords }\OtherTok{\textless{}{-}}\NormalTok{ gbif\_data}\SpecialCharTok{$}\NormalTok{data[ , }\FunctionTok{c}\NormalTok{(}\StringTok{"decimalLongitude"}\NormalTok{, }\StringTok{"decimalLatitude"}\NormalTok{, }\StringTok{"individualCount"}\NormalTok{, }\StringTok{"occurrenceStatus"}\NormalTok{, }\StringTok{"coordinateUncertaintyInMeters"}\NormalTok{, }\StringTok{"institutionCode"}\NormalTok{, }\StringTok{"references"}\NormalTok{)]}

\NormalTok{myspecies\_coords }\OtherTok{\textless{}{-}} \FunctionTok{st\_as\_sf}\NormalTok{(myspecies\_coords, }\AttributeTok{coords =} \FunctionTok{c}\NormalTok{(}\StringTok{"decimalLongitude"}\NormalTok{, }\StringTok{"decimalLatitude"}\NormalTok{), }\AttributeTok{crs =} \FunctionTok{st\_crs}\NormalTok{(}\DecValTok{4326}\NormalTok{))}

\CommentTok{\# CLEAN THE DATASET! {-}{-}{-}{-}}
\CommentTok{\#names(myspecies\_coords)}
\CommentTok{\#sort(unique(myspecies\_coords$individualCount))  \# notice if some points correspond to zero abundance}
\CommentTok{\#sort(unique(myspecies\_coords$occurrenceStatus))  \# check for different indications of "absent", which could be in different languages! and remember that R is case{-}sensitive}
\end{Highlighting}
\end{Shaded}

\hypertarget{additional-data}{%
\subsection{Additional Data}\label{additional-data}}

\begin{Shaded}
\begin{Highlighting}[]
\NormalTok{fgdb }\OtherTok{\textless{}{-}} \StringTok{"C:/Users/rborthwi/OneDrive {-} MADRONE ENVIRONMENTAL SERVICES LTD/Documents/Data Analysis/22.0253/bc{-}cdc{-}species{-}habitat{-}modelling/data/ModelData.gdb/ModelData.gdb"}

\CommentTok{\# List all feature classes in a file geodatabase}
\NormalTok{fc\_list }\OtherTok{\textless{}{-}} \FunctionTok{ogrListLayers}\NormalTok{(fgdb)}
\FunctionTok{print}\NormalTok{(fc\_list)}
\end{Highlighting}
\end{Shaded}

\begin{verbatim}
## [1] "R_PLSK_Range"          "A_ASTR_Range"          "R_PLSK_EO"            
## [4] "R_PLSK_EO_Ext"         "R_PLSK_WSI_SO"         "R_PLSK_iNat_Obsc_True"
## [7] "A_ASTR_WSI_SO"         "A_ASTR_iNat_Obsc_True" "A_ASTR_EO"            
## attr(,"driver")
## [1] "OpenFileGDB"
## attr(,"nlayers")
## [1] 9
\end{verbatim}

\begin{Shaded}
\begin{Highlighting}[]
\NormalTok{\{echo}\OtherTok{=}\ConstantTok{FALSE}\NormalTok{\}}
\CommentTok{\#Second, load the species data of interest {-} write function that automates this:}
\NormalTok{fc\_PLSK\_range }\OtherTok{\textless{}{-}}\NormalTok{ sf}\SpecialCharTok{::}\FunctionTok{st\_read}\NormalTok{(fgdb, }\AttributeTok{layer =}\NormalTok{ fc\_list[[}\DecValTok{1}\NormalTok{]])}
\end{Highlighting}
\end{Shaded}

\begin{verbatim}
## Reading layer `R_PLSK_Range' from data source 
##   `C:\Users\rborthwi\OneDrive - MADRONE ENVIRONMENTAL SERVICES LTD\Documents\Data Analysis\22.0253\bc-cdc-species-habitat-modelling\data\ModelData.gdb\ModelData.gdb' 
##   using driver `OpenFileGDB'
## Simple feature collection with 11 features and 24 fields
## Geometry type: MULTIPOLYGON
## Dimension:     XY
## Bounding box:  xmin: 1372462 ymin: 458775.4 xmax: 1782422 ymax: 701431
## Projected CRS: NAD83 / BC Albers
\end{verbatim}

\begin{Shaded}
\begin{Highlighting}[]
\NormalTok{fc\_PLSK\_EO }\OtherTok{\textless{}{-}}\NormalTok{ sf}\SpecialCharTok{::}\FunctionTok{st\_read}\NormalTok{(fgdb, }\AttributeTok{layer =}\NormalTok{ fc\_list[[}\DecValTok{3}\NormalTok{]])}
\end{Highlighting}
\end{Shaded}

\begin{verbatim}
## Reading layer `R_PLSK_EO' from data source 
##   `C:\Users\rborthwi\OneDrive - MADRONE ENVIRONMENTAL SERVICES LTD\Documents\Data Analysis\22.0253\bc-cdc-species-habitat-modelling\data\ModelData.gdb\ModelData.gdb' 
##   using driver `OpenFileGDB'
## Simple feature collection with 37 features and 55 fields
## Geometry type: MULTIPOLYGON
## Dimension:     XY
## Bounding box:  xmin: 1466241 ymin: 469730.9 xmax: 1697319 ymax: 641128.5
## Projected CRS: NAD83 / BC Albers
\end{verbatim}

\begin{Shaded}
\begin{Highlighting}[]
\NormalTok{fc\_PLSK\_EO\_Ext }\OtherTok{\textless{}{-}}\NormalTok{ sf}\SpecialCharTok{::}\FunctionTok{st\_read}\NormalTok{(fgdb, }\AttributeTok{layer =}\NormalTok{ fc\_list[[}\DecValTok{4}\NormalTok{]])}
\end{Highlighting}
\end{Shaded}

\begin{verbatim}
## Reading layer `R_PLSK_EO_Ext' from data source 
##   `C:\Users\rborthwi\OneDrive - MADRONE ENVIRONMENTAL SERVICES LTD\Documents\Data Analysis\22.0253\bc-cdc-species-habitat-modelling\data\ModelData.gdb\ModelData.gdb' 
##   using driver `OpenFileGDB'
## Simple feature collection with 1 feature and 63 fields
## Geometry type: MULTIPOLYGON
## Dimension:     XY
## Bounding box:  xmin: 1490685 ymin: 672770.7 xmax: 1490735 ymax: 672820.7
## Projected CRS: NAD83 / BC Albers
\end{verbatim}

\begin{Shaded}
\begin{Highlighting}[]
\NormalTok{fc\_PLSK\_WSI\_SO }\OtherTok{\textless{}{-}}\NormalTok{ sf}\SpecialCharTok{::}\FunctionTok{st\_read}\NormalTok{(fgdb, }\AttributeTok{layer =}\NormalTok{ fc\_list[[}\DecValTok{5}\NormalTok{]])}
\end{Highlighting}
\end{Shaded}

\begin{verbatim}
## Reading layer `R_PLSK_WSI_SO' from data source 
##   `C:\Users\rborthwi\OneDrive - MADRONE ENVIRONMENTAL SERVICES LTD\Documents\Data Analysis\22.0253\bc-cdc-species-habitat-modelling\data\ModelData.gdb\ModelData.gdb' 
##   using driver `OpenFileGDB'
## Simple feature collection with 1322 features and 132 fields
## Geometry type: POINT
## Dimension:     XY
## Bounding box:  xmin: 1453130 ymin: 469601.5 xmax: 1697269 ymax: 570309.8
## Projected CRS: NAD83 / BC Albers
\end{verbatim}

\begin{Shaded}
\begin{Highlighting}[]
\NormalTok{fc\_PLSK\_iNat\_obsc }\OtherTok{\textless{}{-}}\NormalTok{ sf}\SpecialCharTok{::}\FunctionTok{st\_read}\NormalTok{(fgdb, }\AttributeTok{layer =}\NormalTok{ fc\_list[[}\DecValTok{6}\NormalTok{]])}
\end{Highlighting}
\end{Shaded}

\begin{verbatim}
## Reading layer `R_PLSK_iNat_Obsc_True' from data source 
##   `C:\Users\rborthwi\OneDrive - MADRONE ENVIRONMENTAL SERVICES LTD\Documents\Data Analysis\22.0253\bc-cdc-species-habitat-modelling\data\ModelData.gdb\ModelData.gdb' 
##   using driver `OpenFileGDB'
## Simple feature collection with 6 features and 50 fields
## Geometry type: POINT
## Dimension:     XY
## Bounding box:  xmin: 1460940 ymin: 508797 xmax: 1607856 ymax: 550244.5
## Projected CRS: NAD83 / BC Albers
\end{verbatim}

\hypertarget{plots}{%
\subsection{Plots}\label{plots}}

Testing the data and visualizing it:

\begin{verbatim}
##       xmin       ymin       xmax       ymax 
## -123.73209   48.99881 -114.05415   52.49719
\end{verbatim}

\includegraphics{00_SpeciesDataLoading_files/figure-latex/pressure-1.pdf}

\hypertarget{generate-the-data-citations}{%
\subsection{Generate The Data
Citations}\label{generate-the-data-citations}}

\begin{verbatim}
## [[1]]
## <<rgbif citation>>
##    Citation: iNaturalist contributors, iNaturalist (2023). iNaturalist
##         Research-grade Observations. iNaturalist.org. Occurrence dataset
##         https://doi.org/10.15468/ab3s5x accessed via GBIF.org on 2023-01-20..
##         Accessed from R via rgbif (https://github.com/ropensci/rgbif) on
##         2023-01-19
##    Rights: http://creativecommons.org/licenses/by-nc/4.0/legalcode
## 
## [[2]]
## <<rgbif citation>>
##    Citation: Wheeler E, MacIntosh H (2019). Royal BC Museum - Herpetology
##         Collection. Royal British Columbia Museum. Occurrence dataset
##         https://doi.org/10.5886/c9s8bp accessed via GBIF.org on 2023-01-20..
##         Accessed from R via rgbif (https://github.com/ropensci/rgbif) on
##         2023-01-19
##    Rights: http://creativecommons.org/licenses/by/4.0/legalcode
## 
## [[3]]
## <<rgbif citation>>
##    Citation: Szabo I (2016). Cowan Tetrapod Collection - Herpetology. Version
##         6.1. University of British Columbia. Occurrence dataset
##         https://doi.org/10.15468/6kf5sp accessed via GBIF.org on 2023-01-20..
##         Accessed from R via rgbif (https://github.com/ropensci/rgbif) on
##         2023-01-19
##    Rights: http://creativecommons.org/publicdomain/zero/1.0/legalcode
## 
## [[4]]
## <<rgbif citation>>
##    Citation: Khidas K, Torgersen J (2023). Canadian Museum of Nature Amphibian
##         and Reptile Collection. Version 1.126. Canadian Museum of Nature.
##         Occurrence dataset https://doi.org/10.15468/hglsas accessed via
##         GBIF.org on 2023-01-20.. Accessed from R via rgbif
##         (https://github.com/ropensci/rgbif) on 2023-01-19
##    Rights: http://creativecommons.org/licenses/by/4.0/legalcode
\end{verbatim}

\end{document}
